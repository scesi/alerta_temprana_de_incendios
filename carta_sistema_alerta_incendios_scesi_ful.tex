\documentclass[11pt,letterpaper]{letter}
\usepackage[spanish]{babel}
\usepackage[utf8]{inputenc}
\usepackage{latexsym,amsmath,amssymb,amsthm}
\usepackage{graphicx}
\usepackage{pifont}
\usepackage[pdftex=true,colorlinks=true,plainpages=false]{hyperref}
\hypersetup{urlcolor=blue}
\hypersetup{linkcolor=black}
\hypersetup{citecolor=black}
\usepackage{lastpage}
\usepackage{url}
\usepackage{anysize}
\marginsize{2cm}{2cm}{0.5cm}{1.7cm}
\usepackage{fancyhdr}
\usepackage{anyfontsize}
\usepackage{tocbibind}
\usepackage{eso-pic}
\usepackage{mathptmx}
\usepackage{draftwatermark}
\usepackage{multirow}
%\usepackage[usenames,dvipsnames]{xcolor}
\usepackage{blindtext}


%-------------------------

\pagestyle{fancy}
\lhead{
}
\chead{
	{ \Huge \tt Universidad Mayor de San Simón }\\
	{ \tt \huge Facultad de Ciencias y Tecnología }\\
	{ \tt \Large Sociedad Científica de Estudiantes de Sistemas e Informática }
}
\rhead{
}
\lfoot{
	{ \scriptsize \tt \url{http://www.scesi.org} } \\
	{ \scriptsize \tt \url{http://www.scesi.memi.umss.edu.bo} }
}
\cfoot{
	{ \scriptsize \tt Prolongación calle Sucre - Parque la Torre - Teléfono: 74312946 } \\
	{ \scriptsize \tt Bloque central de la FCyT - ultimo piso } \\
}
\rfoot{
	{ \small \tt root@scesi.org } \\
 	\thepage/\pageref{LastPage} 
}

\renewcommand{\headrule}{
	{
		\color{brown}
		\hrule width\headwidth height\headrulewidth\vskip-\headrulewidth
	}
}
\renewcommand{\footrule}{
	{
		\color{brown}%
		\vskip-\footruleskip\vskip-\footrulewidth
		\hrule width\headwidth height\footrulewidth\vskip\footruleskip
	}
}
\renewcommand{\headrulewidth}{3pt}
\renewcommand{\footrulewidth}{3pt}
\headheight = 56pt
\headsep = 10pt
\footskip = 28pt
%--------------------------

\newcommand\BackgroundPic{
	\put(502,728){
		\parbox[b][\paperheight]{\paperwidth}{
			\includegraphics[scale=0.9]{../../img/logoSinFondo.png}
		}
	}
	\put(40,729){
		\parbox[b][\paperheight]{\paperwidth}{
			\includegraphics[scale=0.08]{../../img/logoUMSS.jpg}
		}
	}
}

\setlength{\parindent}{0mm}
\setlength{\parskip}{3mm}

%---------------------------
\SetWatermarkAngle{0}
%\SetWatermarkLightness{0.9}
%\SetWatermarkFontSize{1cm}
\SetWatermarkScale{3}
\SetWatermarkText{\includegraphics[scale=0.4]{../../img/waterMark2.png}}
%--------------------------------------

\begin{document}
\AddToShipoutPicture{\BackgroundPic}
%	\tableofcontents


\date{\today}
\begin{letter}{ FUL UMSS \\ \underline {Presente}. -}

%Email: fecortez@hotmail.com
%Celular: 70219848

\begin{center}
	\opening{ \textbf{REF.: SOLICITUD DE FINANCIAMIENTO PARA PROYECTO ``SISTEMA DE ALERTA TEMPRANA DE INCENDIOS FORESTALES" } }
\end{center}

De nuestra consideración:

A tiempo de enviarle un cordial saludo y desearle éxito en las funciones que realiza, aprovecho la oportunidad para hacerle conocer que los estudiantes integrantes de la Sociedad Científica de Estudiantes de Sistemas e Informática conjuntamente con la dirección de la carrera de Informática, estamos realizando un proyecto de ``Creación de un sistema de alerta temprana de incendios forestales utilizando drones de bajo costo".\\

El proyecto se llevara a cabo desde el dia sabado 31 de enero al miercoles 4 de marzo del 2015 (33 días).

En ese sentido le hacemos saber que para llevar a cabo el proyecto, se necesita pagar el viaje y estadía durante el proceso de desarrollo del proyecto, del  experto {\bf PhD. Eduardo Di Santi}, ya que tiene un amplio conocimiento con el desarrollo de proyectos colaborativos y manejo de grupos de investigacion. Ademas de que tiene dominio con el tema de sensores y aeromodelismo.\\

Para solicitar el financiamiento en primera instancia se elaboro el proyecto como propuesta hacia el Dr. Di Santi, obteniendo una respuesta modificada y detallada en el proceso de desarrollo del proyecto denominado {\it Taller de ingeniería de software}, Como también se fue cotizando los pasajes de viaje en avión del Dr. Di Santi desde Southampton - Cbba y de vuelta de Cbba - Southampton.\\

Para respaldar nuestra solicitud, junto con esta carta adjuntamos:
\begin{itemize}
\item Proyecto Elaborado por parte de la SCESI y Dirección de Carrera de Ingeniería Informática.
\item Respuesta al proyecto por parte del PhD. Eduardo Di Santi.
\item Presupuesto de Pasajes y viáticos para el PhD. Eduardo Di Santi.
\item Hoja de Vida del PhD. Eduaro Di Santi. Para dar credibilidad a su amplia trayectoria de investigación.
\end{itemize}

Sin otro particular y a la espera de una respuesta favorable a esta solicitud, me despido con las consideraciones más distinguidas.

Atentamente,

\vspace{0.50cm}

\begin{center}
...................................\\
Ubaldino Zurita\\
{\bfseries Presidente de la SCESI}
\end{center}
%\vspace{0.1cm}
cc Archivo.
\end{letter}

\end{document}
