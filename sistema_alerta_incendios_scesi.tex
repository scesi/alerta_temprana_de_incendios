\documentclass[letter,12pt]{article}

\usepackage[T1]{fontenc}
\usepackage{lmodern}
\renewcommand*\familydefault{\sfdefault} %% Only if the base font of the document is to be sans serif

\usepackage[spanish]{babel}
\usepackage[utf8]{inputenc}
\usepackage[pdftex]{graphicx}
%\usepackage{pifont} %Listas con dedo
\usepackage{hyperref}
%\usepackage{url}
\usepackage{fancyhdr}
\usepackage{lastpage}
\pagestyle{fancy} % Cabeceras/Pies
\fancypagestyle{plain}% Para la primera página
{%
         \fancyhead[l]{\includegraphics[width=0.1\textwidth]{../img/logoUMSS.png}}
         \fancyhead[r]{\includegraphics[width=0.13\textwidth]{../img/fcyt_logo.png}}
         %REMPLAZAR <NombreDelProyecto> POR EL NOMBRE DE TU PROYECTO
         \fancyhead[c]{}
         \renewcommand{\headrulewidth}{0.5pt}
         \fancyfoot[l]{Dirección de carrera Ingeniería Informática \\ Sociedad Cient\'ifica de Estudiantes de Sistemas e Inform\'atica}
         \fancyfoot[c]{}
         %\fancyfoot[r]{\thepage/\pageref{LastPage}}
         \renewcommand{\footrulewidth}{0.5pt}
}
% Para el resto de páginas


% % %   1 de febrero - 5 marzo

\lhead{Creación de un sistema de alerta temprana de incendios}
%REMPLAZAR <NombreDelProyecto> POR EL NOMBRE DE SU PROYECTO
\chead{}
\rhead{\includegraphics[width=0.1\textwidth]{../img/logo2.png}}
\renewcommand{\headrulewidth}{0.4pt}
\lfoot{Dirección de carrera Ingeniería Informática \\ Sociedad Cient\'ifica de Estudiantes de Sistemas e Inform\'atica \\ 
\url {http://scesi.memi.umss.edu.bo}}
\cfoot{}
\rfoot{\thepage/\pageref{LastPage}}
\renewcommand{\footrulewidth}{0.4pt}
% Title Page
%REMPLAZAR <NombreDelProyecto> POR EL NOMBRE DE SU PROYECTO
\title{Proyecto\\ Creación de un sistema de alerta temprana de incendios}
%NOMBRE DEL AUTOR O AUTORES
\author{Ubaldino Zurita\\{\normalsize Presidente SCESI}\\\\Mgr. Lic. Rosemary Torrico Bascopé\\{\normalsize Directora de carrera Ingeniería Informática}}


\begin{document}
\maketitle
\begin{center}
    \includegraphics[width=1.0\textwidth]{../img/scesi_informatica_logo.png} 
\end{center}
\begin{center}
    \url {http://scesi.memi.umss.edu.bo}\\
\end{center}
\begin{center}
    \url{http://www.cs.umss.edu.bo/rep_carrera.jsp?codigo=1}
\end{center}
\pagebreak
\tableofcontents
\pagebreak
\begin{abstract}
el proyecto propuesto se trata de crear un taller de ingeniería de software, que requerirá de la inclusión de varios estudiantes, investigadores, docentes e instituciones. Para crear un prototipo de un \underline{sistema de alerta} \underline{temprana de incendios}, para poder utilizarla en los bosques forestales que sufren de incendios y que con este sistema se tiene una reacción rápida para mitigar el gran daño que estos incendios causan al medio ambiente.\\
Este taller esta pensado para fortalecer el trabajo colaborativo entre grupos que se encargaran a realizar tareas especificas para el proyecto, que se irán integrando conforme se vaya desarrollando el proyecto.
\end{abstract}
\pagebreak

\section{Introducci\'on}
En varias partes del mundo los bosques son cuidados con sistemas alertan de incendios, invasiones, derrumbes o cualquier otra catástrofe. Estos sistemas ayudan a controlar de forma rápida cualquier catástrofe que los bosques lleguen a sufrir, ademas de que también se pueden generar patrones para ya tener un plan de acción según a la catástrofe suscitada.\\
El origen de la creación de estos sistemas es siempre la necesidad de poder realizar un plan de acción apenas los bosques sufran una catástrofe, y en este caso los incendios forestales.
\section{Antecedentes}
Los incendios forestales que cada día se suscita en varias regiones de los bosques en el mundo, hacen que se pierda un parte de la vegetación perteneciente a todo un ecosistema. Es por esto que se decide realizar un sistema de \underline{sistema de alerta temprana de incendios} con el fin de mitigar la rápida propagación que estos incendios llegan a tener debido a los vientos, calor excesivo. 
\section{Definici\'on del Problema}
En cochabamba mas propiamente en el Parque Nacional Tunari, se producen incendios repentinos y muy a menudo que a falta de un sistema de alerta temprana que llegue a comunicar el lugar, magnitud , dirección, etc. a las instancias que mitigan los incendios que son los bomberos, se pierde gran cantidad de forestación que daña todo un ecosistema que se forma en el área. Esto afecta también a los alrededores que son los afectados indirectos.\\
\section{Objetivo General}
Aprender a crear un sistema colaborativo que incluya a profesionales, estudiantes , activistas, investigadores ademas de instituciones. Con el fin de realizar un aporte útil a la sociedad en la que vivimos.\\
Con la creación de un sistema de monitoreo de incendios forestales, que incluyan sensores , drones , camaras como mínimo. Todo esto para alertar a las instancias correspondientes que están a cargo del cuidado de los bosques forestales.
\section{Objetivos Espec\'ificos}
creación robusta del sistema, de modo que sea útil y aplicable a cualquier medio.\\
Realizar trabajos multidisciplinarios para fomentar el trabajo de varios grupos en el proyecto.\\
Realizar trabajos de campo, con el objetivo de realizar estudios sobre la comunicación de estos dispositivos que se irán trabajando por los distintos grupos conformados.
\section{Recursos}
\begin{itemize}
\item Instructores con capacidad de investigación y manejo de grupos de trabajo, para poder guiar el proceso de desarrollo del proyecto.
\item Drones Aéreos de prueba.
\item Dispositivos móviles con el sistema Android.
\item Red wifi, para la comunicación.
\item Computadoras para el desarrollo.
\end{itemize}

%\section{Herramientas}

%\section{Método o Técnica}

\section{Justificaci\'on}
Este proyecto se decide realizar con el fin de que nuestros parques que son el pulmón de nuestras zonas, lleguen a ser monitorizadas para que apenas suceda un incendio este sistema realice un monitoreo mediante sensores y drones, para luego analizar la causa y el lugar aproximado de origen, ademas de la dirección de propagación.\\
Este sistema va a comunicar al cuerpo de bomberos mas próximo al lugar para que estos tengan conocimiento y realicen un plan de acción para mitigar el incendio ya sabiendo las coordenadas exactas y el tipo de bosque.
\end{document}          
