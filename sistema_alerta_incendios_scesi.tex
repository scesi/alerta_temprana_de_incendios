\documentclass[letter,12pt]{article}

\usepackage[T1]{fontenc}
\usepackage{lmodern}
\renewcommand*\familydefault{\sfdefault} %% Only if the base font of the document is to be sans serif

\usepackage[spanish]{babel}
\usepackage[utf8]{inputenc}
\usepackage[pdftex]{graphicx}
%\usepackage{pifont} %Listas con dedo
\usepackage{hyperref}
%\usepackage{url}
\usepackage{fancyhdr}
\usepackage{lastpage}
\pagestyle{fancy} % Cabeceras/Pies
\fancypagestyle{plain}% Para la primera página
{%
         \fancyhead[l]{\includegraphics[width=0.1\textwidth]{../img/logoUMSS.png}}
         \fancyhead[r]{\includegraphics[width=0.13\textwidth]{../img/fcyt_logo.png}}
         %REMPLAZAR <NombreDelProyecto> POR EL NOMBRE DE TU PROYECTO
         \fancyhead[c]{}
         \renewcommand{\headrulewidth}{0.5pt}
         \fancyfoot[l]{Dirección de carrera Ingeniería Informática \\ Sociedad Cient\'ifica de Estudiantes de Sistemas e Inform\'atica}
         \fancyfoot[c]{}
         %\fancyfoot[r]{\thepage/\pageref{LastPage}}
         \renewcommand{\footrulewidth}{0.5pt}
}
% Para el resto de páginas


% % %   1 de febrero - 5 marzo

\lhead{Creación de un sistema de alerta temprana de incendios}
%REMPLAZAR <NombreDelProyecto> POR EL NOMBRE DE SU PROYECTO
\chead{}
\rhead{\includegraphics[width=0.1\textwidth]{../img/logo2.png}}
\renewcommand{\headrulewidth}{0.4pt}
\lfoot{Dirección de carrera Ingeniería Informática \\ Sociedad Cient\'ifica de Estudiantes de Sistemas e Inform\'atica \\ 
\url {http://scesi.memi.umss.edu.bo}}
\cfoot{}
\rfoot{\thepage/\pageref{LastPage}}
\renewcommand{\footrulewidth}{0.4pt}
% Title Page
%REMPLAZAR <NombreDelProyecto> POR EL NOMBRE DE SU PROYECTO
\title{Proyecto\\ Creación de un sistema de alerta temprana de incendios}
%NOMBRE DEL AUTOR O AUTORES
\author{Ubaldino Zurita\\{\normalsize Presidente SCESI}\\\\Mgr. Lic. Rosemary Torrico Bascopé\\{\normalsize Directora de carrera Ingeniería Informática}}


\begin{document}
\maketitle
\begin{center}
    \includegraphics[width=1.0\textwidth]{../img/scesi_informatica_logo.png} 
\end{center}
\begin{center}
    \url {http://scesi.memi.umss.edu.bo}\\
\end{center}
\begin{center}
    \url{http://www.cs.umss.edu.bo/rep_carrera.jsp?codigo=1}
\end{center}
\pagebreak
\tableofcontents
\pagebreak
\begin{abstract}
El proyecto propuesto se trata de crear un taller de ingeniería de software, que requerirá de la inclusión de varios estudiantes, investigadores, docentes e instituciones. Para crear un prototipo de un \underline{sistema de alerta} \underline{temprana de incendios}, para poder utilizarla en los bosques forestales que sufren de incendios y que con este sistema se pueda obtener una reacción rápida para mitigar el gran daño que estos incendios causan al medio ambiente.\\

Este taller esta pensado para fortalecer el trabajo colaborativo entre grupos que se encargaran a realizar tareas especificas para el proyecto, que se irán integrando conforme se vaya desarrollando el proyecto.\\

Es en este sentido que también esta orientado para crear una propuesta de vigilancia específicamente al territorio de nuestro parte Tunari, que actualmente esta sufriendo de reducción de áreas forestales y recursos hídricos que solventan a la ciudad de Cochabamba con el agua potable.\\

Para hacer posible este proyecto se logro contactar el {\bf Phd. Eduardo Di Santi}, experto en el tema de los vehículos no tripulados y desarrollo de sistemas informáticos. Que encarecidamente nos envió una propuesta de trabajo durante todo el proyecto.\\

\end{abstract}
\pagebreak

\section{Introducci\'on}
En varias partes del mundo los bosques son cuidados con sistemas que alertan de incendios, invasiones, derrumbes o cualquier otra catástrofe. Estos sistemas ayudan a controlar de forma rápida cualquier catástrofe que los bosques lleguen a sufrir, ademas que también se pueden generar patrones para ya tener un plan de acción según a la catástrofe suscitada.\\

El origen de la creación de estos sistemas es siempre la necesidad de poder realizar un plan de acción apenas los bosques sufran una catástrofe, y en este caso los incendios forestales.
\section{Antecedentes}
Los incendios forestales que cada día se suscita en varias regiones de los bosques en el mundo, hacen que se pierda un parte de la vegetación perteneciente a todo un ecosistema. Es por esto que se decide realizar un sistema de \underline{sistema de alerta temprana de incendios} con el fin de mitigar la rápida propagación que estos incendios llegan a tener debido a los vientos, calor excesivo y en algunos casos factor humano. 
\section{Definici\'on del Problema}
En Cochabamba mas propiamente en el Parque Nacional Tunari, se producen incendios repentinos y muy a menudo que a falta de un sistema de alerta temprana que llegue a comunicar el lugar, magnitud y dirección del incendio a las instancias que mitigan los incendios que son los bomberos, se pierde gran cantidad de forestación que daña todo un ecosistema que se forma en el área. Esto afecta también a los alrededores del parque.\\

Estos incendios surgen por las altas temperaturas de calor que llegan a ocasionar alguna chispa en los lugares mas secos del parque que comienzan a propagarse hasta ocasionar una ola de incendios de gran magnitud que suelen ser alimentados por los vientos. Pero muchas veces se tomo denuncias de que estos incendios también son provocados por personas que están en un constante acecho al asentamiento sobre estas tierras que posee el parque Tunari, de tal manera que no les importa ocasionar daños a la forestación y los recursos hídricos del lugar.\\

Actualmente el parque Tunari no cuenta con una cantidad suficiente de guardabosques y peor aun no se cuenta con herramientas de monitoreo del área, esto hace que no se tenga una reacción rápida a los focos de incendios producidos. 

\section{Objetivo General}
Aprender a crear un sistema colaborativo que incluya a profesionales, estudiantes, activistas, investigadores ademas de instituciones. Con el fin de realizar un aporte útil a la sociedad en la que vivimos.\\
Con la creación de un sistema de monitoreo de incendios forestales que incluyan sensores, drones y cámaras filmadoras como mínimo. Todo esto para alertar a las instancias correspondientes que están a cargo del cuidado de los bosques forestales.\\

\section{Objetivos Espec\'ificos}
\begin{itemize}
\item Realizar trabajos multidisciplinarios para fomentar el trabajo de varios grupos en el proyecto.
\item Realizar trabajos de campo, con el objetivo de realizar estudios sobre la comunicación de estos dispositivos que se irán trabajando por los distintos grupos conformados.
\item Generar una herramienta de monitoreo y comunicación con las instancias y organizaciones que puedan ayudar con la mitigación rápida de estos incendios.
\item Crear un grupo especializado para continuar con este proyecto para futuras implementaciones, en parques que quisieran adquirir este sistema de monitoreo y también como soporte técnico para que este proyecto no se quede solo con la construcción sino que también con el pasar del tiempo se vaya refactorizando el sistema y los equipos de monitoreo.
\end{itemize}


\section{Recursos}
\begin{description}
\item[Instructores con capacidad de investigación]~\\
Los instructores se encargaran de coordinar aspectos técnicos y de desarrollo del proyecto con el Phd. Eduardo Di Santi para que que una vez terminado el proyecto estos instructores continúen con los futuros enprendimientos de este proyecto y puedan transmitir los conocimientos adquiridos a los nuevos integrantes al proyecto.
\item[Drones Aéreos de prueba.]~\\
Los Drones Aéreos nos servirán para el monitoreo de las áreas vigiladas por los guardabosques que al incorporar camaras filmadoras hace que se puedan obtener imágenes en poco tiempo de las zonas en peligro. 
\item[Dispositivos móviles con el sistema Android.]~\\
Los dispositivos móviles nos servirán para interaccionar con estos drones aéreos y algunos sensores de calor, humo integrados al sistema.
\item[Red wifi, para la comunicación]~\\
La red wifi vamos a utilizarla como el protocolo de comunicación para comandar a los drones desde los dispositivos móviles.
\item[Computadoras para el desarrollo]~\\
Las computadoras nos serán útiles durante toda la fase de desarrollo y pruebas ya se desarrollará un sistema de monitoreo.
\end{description}

%\section{Herramientas}

%\section{Método o Técnica}

\section{Justificaci\'on}
Este proyecto se decide realizar con el fin de que nuestros parques que son el pulmón de nuestras zonas, lleguen a ser monitorizadas para que apenas suceda un incendio este sistema realice un monitoreo mediante sensores y los drones, para luego analizar la causa y el lugar aproximado de origen, ademas de la dirección de propagación.\\

Este sistema va a comunicar al cuerpo de bomberos mas próximo al lugar para que estos tengan conocimiento y realicen un plan de acción para mitigar el incendio ya sabiendo las coordenadas exactas y el tipo de bosque.\\

Este primer aporte en el área de monitoreo de bosques como Universidad Mayor de San Simón, consideramos que es importante para nuestra población de Cochabamba ya que tenemos el parque Tunari que necesita de este sistema para no sufrir una catástrofe cada vez que se origine un incendio. 
\end{document}          
